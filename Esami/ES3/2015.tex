\begin{figure}[h!]
\textbf{Tema d'Esame di Gennaio 2015}\\ \\
Calcolare il valore dell'accellerazione di gravità alla superfice del pianeta Venere, sapendo che la sua velocità di fuga vale $10.36km/s$ e il raggio è di $6052km$\\ \\
    \noindent\fbox{
		\parbox{\textwidth}{
			\null\hfill \textbf{Soluzione:} $a_{gLuna}=8.86m/s^2$\\
			\textbf{Procedimento: } \\
            Trasformare tutte le unità di misura secondo le convenzioni del Sistema Internazionale.\\
            $v_{fLuna}=10.36km/s=10.36\cdot10^3m/s$\\
            $r_{luna}=6052km=6052\cdot10^3 m$\\ \\
            Sapendo che:
            \begin{gather*}
                a_g=\frac{M_{Luna}\cdot G}{r_{Luna}^2}  \qquad  \qquad v_f=\sqrt{\frac{2\cdot G \cdot M_{Luna}}{r_{luna}}}  \\
                G\cdot M_{Luna}=\frac{v_{luna}^2 \cdot r_{luna}}{2}= \frac{(10.36\cdot10^3m/s)^2 \cdot 6052\cdot10^3 m}{2}=3.2478 \cdot 10^{14}m^3/s^2\\
                a_{g_{Luna}}=\frac{G\cdot M_{Luna}}{r_{luna}^2}=\frac{3.2478 \cdot 10^{14}m^3/s^2}{(6052\cdot 10^3m)^2}=8.86m/s^2 
            \end{gather*}
            
		}
	}	
\end{figure}

\begin{figure}[h!]
\textbf{Tema d'Esame di Febbraio 2015}\\ \\
Calcolare il periodo di rotazione della Luna attorno alla Terra assumendo che percorra un'orbita circolare di raggio $384000 km$, conoscendo l'accelerazione di gravità sulla superficie della Terra, $g = 9.8 m/s^2$ e il raggio della Terra $6370 km$. \\ \\
\noindent\fbox{
    \parbox{\textwidth}{
        \null\hfill \textbf{Soluzione:} $T=27d:10h:36min $\\
        \textbf{Procedimento: } \\
        Trasformare tutte le unità di misura secondo le convenzioni del Sistema Internazionale.\\
        $r_{orb}=384000km = 3.84 \cdot 10^8m$\\
        $r_{terra}=6370km=6.370\cdot 10^6 m$\\ \\
        Sapendo che:
        \begin{gather*}
            a_g=\frac{M_{Terra}\cdot G}{r_{Terra}^2}  \qquad  \qquad T^2=\frac{4\cdot \pi^2\cdot r_{orb}^3}{G\cdot M}  \\
           G\cdot M=a_g\cdot r_t^2 =9.8m/s^2 \cdot (6.370\cdot 10^6 m)^2=3.9765 \cdot 10^{14} m^3/s^2\\
           T=\sqrt{\frac{4\cdot \pi^2\cdot r_{orb}^3}{G\cdot M}}=\sqrt{\frac{4\cdot \pi^2\cdot (3.84 \cdot 10^8m)^3}{3.9765 \cdot 10^{14} m^3/s^2}}=5.6214\cdot \sqrt{10^{12}s^2}=2370960.041s\\
           =27d:10h:36min 
        \end{gather*}
        
    }
}	
\end{figure}

\begin{figure}[h!]
\textbf{Tema d'Esame di Giugno 2015}\\ \\
Qual'è la velocità di fuga da un asteroide (sferico) di raggio $800km$ e per il quale l'accellerazione di gravità sulla superfice vale $6m/s^2$?\\ \\
\noindent\fbox{
    \parbox{\textwidth}{
        \null\hfill \textbf{Soluzione:} $ v_f=3098.386m/s $\\
        \textbf{Procedimento: } \\
        Trasformare tutte le unità di misura secondo le convenzioni del Sistema Internazionale.\\
        $r_{orb}=384000km = 3.84 \cdot 10^8m$\\
        $r_{terra}=6370km=6.370\cdot 10^6 m$\\ \\
        Sapendo che:
        \begin{gather*}
            a_g=\frac{M_{Asteroide}\cdot G}{r_{Asteroide}^2}  \qquad  \qquad v_f=\sqrt{\frac{2\cdot G \cdot M_{Asteroide}}{r_{Asteroide}}} \\
            v_f=\sqrt{2\cdot a_g \cdot r_{Asteroide}}=\sqrt{2\cdot 6m/s^2 \cdot 800 000m}=\sqrt{9600 000m^2/s^2}=3098.386m/s
        \end{gather*}
        
    }
}	
\end{figure}

\begin{figure}[h!]
\textbf{Tema d'Esame di Luglio 2015}\\ \\
Calcolare il periodo orbitale del Telescopio Spaziale Hubble (HST), che compie orbite circolari intorno alla Terra alla quota di $600km$, il raggio della Terra è $6378km$ e la massa $5.98\cdot 10^{24}kg$
\end{figure}