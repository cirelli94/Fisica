\begin{figure}[h!]
    \textbf{Tema d'Esame di Gennaio 2017}\\ \\
    Si ha la necessità di far fuoriuscire dell'acqua (densità $1000 kg/m^3$
    ) contenuta all'interno di una siringa senza ago, posta in orizzontale, alla velocità di $15 cm/s$.
     Stabilire quale differenza di pressione bisogna esercitare tra lo stantuffo e il beccuccio da cui fuoriesce il
    fluido, sapendo che il rapporto tra le due sezioni vale $20$. 
\end{figure}

\begin{figure}[h!]
    \textbf{Tema d'Esame di Febbraio 2017}\\ \\
    Un torchio idraulico è costituito da due vasi cilindrici comunicanti tra loro e contenenti acqua, disposti verticalmente, di sezioni $S_A =2 dm^2$  e $S_B = 10 dm^2$ , rispettivamente. Dentro i vasi possono scorrere, a tenuta e senza attrito, due pistoni $A$ e $B$ di masse $m_A = 20 kg$ e $m_B = 150 kg$. Si calcoli la massa m del carico che si deve porre sul pistone A per ottenere livelli uguali nei due vasi.
\end{figure}