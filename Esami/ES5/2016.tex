\begin{figure}[h!]
\textbf{Tema d'Esame di Gennaio 2016}\\ \\
 Prima di chiudere l'interruttore $S1$, la tensione ai capi del condensatore $C1$ è pari a $12V$. Determinare quanto tempo deve passare dalla chiusura dell'interruttore $S1$ perché la corrente
che scorre in $R2$ diventi inferiore a $10 \mu A$. $R1=R2=2k\Omega$. $C1=1\mu F$.
\begin{center}
		\includegraphics[scale=0.8]{ES5/GEN052016.jpg}
	\end{center}
\end{figure}

\begin{figure}[h!]
\textbf{Tema d'Esame di Febbraio 2016}\\ \\
Si determini la differenza di potenziale ai capi della resistenza $R4$ del circuito mostrato in figura. La differenza di potenziale fornita dalla batteria è di $12V$ e i valori delle resistenze sono rispettivamente $R2=15\Omega, R3=40\Omega,R4=25\Omega, R5=R6=32\Omega, R1=R7=18\Omega$
\begin{center}
		\includegraphics[scale=1.1]{ES5/FEB052016.jpg}
	\end{center}
\end{figure}

\begin{figure}[h!]
\textbf{Tema d'Esame di Giugno 2016}\\ \\
Se il generatore fornisce una differenza di potenziale di $12 V$, qual'è la caduta di potenziale ai capi della resistenza $R5$ ?
$R1= 35 \Omega, R2= 10 \Omega, R3= 24 \Omega, R4= 18 \Omega, R5= 30 \Omega, R6= 21 \Omega, R7=17 \Omega , R8=19 \Omega$.
\begin{center}
		\includegraphics[scale=1]{ES5/GIU052016.jpg}
	\end{center}
\end{figure}

\begin{figure}[h!]
\textbf{Tema d'Esame di Luglio 2016}\\ \\
Nel circuito in figura, la corrente attraverso $R6$ è $i_6=1.2 A$ e le resistenze sono $R1=R2=R3=4.0 \Omega, R4= 10 \Omega, R5= 4.0 \Omega , R6= 2.0 \Omega$. Qual'è la forza elettromotrice della batteria (ideale)?
\begin{center}
		\includegraphics[scale=0.8]{ES5/LUG052016.jpg}
	\end{center}
\end{figure}