\documentclass{article}

\usepackage[utf8]{inputenc}

\usepackage{multicol} % separazione delle colonne
\usepackage[document]{ragged2e}
\usepackage{graphicx}
\usepackage{tabularx}
\usepackage{cancel}
\usepackage{amsmath}
\usepackage[a4paper, total={20cm, 25cm}]{geometry} % margini della pagina

\allowdisplaybreaks % Fix per gli spazi bianchi giganteschi
\setlength{\columnsep}{1cm}




\begin{document}

\title{Formulario di Fisica}
\author{}
\date{}
\maketitle

\begin{multicols}{3}

\section{Varie} 

\begin{gather*}
    \tan (\alpha) = \frac{\sin \alpha}{\cos \alpha} \\
    \cot (\alpha) = \frac{\cos \alpha}{\sin \alpha} \\
    \sin (\alpha) \cos (\alpha) = \sin (2 \alpha) \\
    \textbf{Cerchio} \\
        Area = r^2 \cdot \pi \\
        Circonferenza = \pi \cdot r \\
    \textbf{Sfera} \\
        Volume = \frac{4}{3} \pi r^3 \\
        Superficie = 4 \cdot \pi \cdot r^2 \\
\end{gather*}

\begin{center}
    \includegraphics[width=0.6\linewidth]{Dinamica/piano-inclinato-senza-angolo.png} \\    
\end{center}

\begin{gather*}
    \begin{cases}
        \sin \alpha = \frac{h}{l} \\ 
        \cos \alpha = \frac{d}{l} \\
        \tan \alpha = \frac{h}{d} \\
    \end{cases} \rightarrow 
    \begin{cases}
        h = l \cdot \sin \alpha \\
        l = \frac{h}{\sin \alpha}
    \end{cases} \\
\end{gather*}


\section{Unità di misura}
\begin{center}
  \begin{tabularx}{\textwidth}{ l l }
      Forza & $1 \cdot N = 1 \cdot kg \cdot 1 \frac{m}{s^2}$ \\
      Velocità & $\begin{cases}
        1 \cdot \frac{m}{s} = 3.6 \cdot \frac{km}{h} \\
        1 \cdot \frac{km}{h} = \frac{1}{3.6} \cdot \frac{m}{s}
      \end{cases}$ \\
      Costante Elastica & $\!\begin{aligned}[t]1 \cdot k = 1 \cdot \frac{N}{cm} \\ = 100 \cdot \frac{N}{m}\end{aligned}$ \\
      Oscillazione & $1 \cdot T = 1 \cdot s$ \\
      Lavoro/Energia & $\!\begin{aligned}[t]1 \cdot J = 1 \cdot N \cdot 1 \cdot m \\ = 1 \cdot kg \cdot \frac{m^2}{s^2} \end{aligned}$ \\
      Densità & $\begin{cases}
        1 \cdot \rho = 1 \cdot \frac{kg}{m^3} \\
        1 \cdot \frac{kg}{m^3} = \frac{1}{1000} \cdot \frac{g}{cm^3} \\
        1 \cdot \frac{g}{cm^3} = 1000 \cdot \frac{kg}{m^3}
      \end{cases}$ \\
      Pressione & $1 \cdot Pa = 1 \cdot \frac{N}{m^2}$ \\
      Portata & $1 \cdot Q = 1 \cdot \frac{m^3}{s}$ \\
      Resistenza & $\Omega$ $Ohm$  \\
      Intensità di Corrente & $Ampere$ \\
      Differenza di Potenziale & $Volt$
  \end{tabularx}
\end{center}


\section{Vettori}

\begin{gather*}
    \vec{v} = \begin{cases}
        v_x = v \cdot \cos \alpha \\
        v_y = v \cdot \sin \alpha
    \end{cases} \\
    \vec{v} = \vec{v}_x + \vec{v}_y \\
    v = \sqrt{v_x^2 + v_y^2} \\
    \tan \theta = \frac{v_y}{v_x} \\
    \theta = \arctan (\frac{v_y}{v_x})
\end{gather*}

\section{Costanti}

\begin{gather*}
    \textbf{Forza di gravità: } \\ g_{Terra} = 9.81 \frac{m}{s^2} = 9.81 \frac{N}{kg}\\
    \textbf{Forza di gravità lunare: } \\ g_{Luna} = 1.62 \frac{m}{s^2} \\
    \textbf{Costante Gravitazionale: } \\
    G = 6.67 \cdot 10^{-11} N \cdot \frac{m^2}{kg^2} \\
    \textbf{Raggio Terra: } \\ r_{Terra} = 6370 km \\
    \textbf{Massa Terra: } \\ M_{Terra} = 5.972 \cdot 10^{24} kg \\
    \textbf{Densità: } \\
    \rho_{H_2O} = 1000 \frac{kg}{m^3} \\
    \rho_{aria} = 1.225 \frac{kg}{m^3} \\
    \rho_{terra} = 5.5 \frac{g}{cm^3} \\
    \textbf{Pressione: } \\
    1 \text{ atm} = 101 325 \text{ Pa} \\
    1 \text{ bar} = 100 000 \text{ Pa} = 10^5 \text{ Pa} \\
    1 \text{ atm} \simeq 1.01325 \text{ bar} \\
    \textbf{Costante di Avogadro: } \\
    6.02214086 \cdot 10^{23} mol^{-1} \\
    \textbf{Volume molare: } V_m = 22,414 \ell
\end{gather*}


\section{Cinematica}
%%
\subsection{Moto rettilineo}
\begin{gather*}
    \textbf{Variazione di velocità: } \Delta v = v - v_0 \\
    \textbf{Tempo trascorso: } \Delta t = t - t_0 \\
    \textbf{Distanza percorsa: } \Delta s = \left| s - s_0 \right|
\end{gather*}
\subsection{Moto accelerato}
\textbf{Nota bene: } questo sistema si usa spesso e volentieri per il moto uniforme
\begin{gather*}
    \begin{cases}
        v = v_0 + a \cdot t \\
        s = s_0 + v_0 t + \frac{1}{2} a t^2
    \end{cases}
    \\
    \textbf{Accelerazione:} \\
    \begin{cases}
        a = g = 9.81 m/s^2   & \text{Caduta libera}  \\
        a = -g = - 9.81m/s^2 & \text{Lancio in alto}
    \end{cases}
    \\
    \textbf{Velocità: } v = v_0 + a t \\
    \textbf{Tempo: } t = \frac{v - v_0}{a} \\
    \textbf{Accelerazione: } a = \frac{v - v_0}{t} \\
    \textbf{Accelerazione: } a = \frac{2 (s - s_0 - v_0 t)}{t^2} \\
    \textbf{Velocità(senza t): } \\ v = \sqrt{v_0^2 + 2 a (s - s_0)} \\
    \textbf{Posizione: } s = s_0 + v_0 t + \frac{1}{2} a t^2 \\
    \textbf{Posizione(senza a): } s = s_0 + \frac{v + v_0}{2} t \\
    \textbf{Velocità(senza a): } v = \frac{2 (s - s_0)}{t} - v_0
\end{gather*}
%%%
\subsection{Moto dei proiettili}
\textbf{Nota bene: } l'accelerazione($g$)è negativa quando si presume di partire dal basso verso l'alto, perché la gravita agisce contro il movimento verticale. \\ Al contrario, se ci troviamo in un movimento che parte dall'alto verso il basso, l'accelerazione($g$)sarà positiva!
\begin{gather*}
    \textbf{Equazione parabola: } \\ \begin{cases}
        x = x_0 + v_{0x} t \\
        y = y_0 + v_{0y} t - \frac{1}{2} g t^2
    \end{cases}
    \\
    \textbf{Equazione parabola: } \\ y = \frac{v_x}{v_y} x - \frac{1}{2} \frac{g}{v_x^2} x^2 \\
    \textbf{Velocità iniziale: } v_0 = \sqrt{v_x^2 + v_y^2} \\
    \textbf{Velocità iniziale: } \\ \begin{cases}
        v_{0x} = v_0 \cos \alpha  = v_y \frac{\cos \alpha}{\sin \alpha} \\
        v_{0y} = v_0 \sin \alpha = v_x \frac{\sin \alpha }{\cos \alpha}
    \end{cases}
    \\
    \textbf{Velocità(tempo t): } \begin{cases}
        v_x = v_0 \cos \alpha \\
        v_y = v_0 \sin \alpha - a t
    \end{cases}
    \\
    \textbf{Vertice: } \begin{cases}
        x_v = \frac{v_x v_y}{g} = \frac{v_0^2 \sin (2\alpha)}{g} \\
        y_v = \frac{v_y^2}{2g}
    \end{cases}
    \\
    \textbf{Gittata: } \frac{v_x v_y}{\frac{1}{2}g} = \frac{v_0^2 \sin (2 \alpha)}{\frac{1}{2}g} \\
    \textbf{Tempo di volo: } t = \frac{v_0 \sin \alpha}{2g} \\
    \textbf{Altezza massima: } y = \frac{v_0^2 \sin (2 \alpha)}{2g}
\end{gather*}

%%%
\subsection{Moto circolare}
\begin{center}
    \includegraphics[width=0.8\linewidth]{Cinematica/circonferenza-goniometrica.png}
\end{center}
%%%%
\begin{center}
    \begin{tabular}{ c c c }
        Grandezze     & Lineari & Angolari \\
        Posizione     & $s$     & $\theta$ \\
        Velocità      & $v$     & $\omega$ \\
        Accelerazione & $a$     & $\alpha$ \\
    \end{tabular}
\end{center}
\textbf{Nota bene: } come la tabella sopra ci indica, c'è una corrispondenza tra grandezze lineari e grandezza angolari. \\ Ciò significa che possiamo immaginare il punto che si muove sulla circonferenza come se si muovesse su una retta(la circonferenza \textit{spiaccicata}), e di conseguenza utilizzare le formule del moto rettilineo per trovarne la posizione!
\begin{gather*}
    \textbf{Velocità: } v = v_0 + a t \\
    \textbf{Tempo: } t = \frac{v - v_0}{a} \\
    \textbf{Accelerazione: } a = \frac{v - v_0}{t} \\
    \textbf{Posizione: } s = s_0 + v_0 t + \frac{1}{2} a t
\end{gather*}
\subsubsection{Moto circolare uniforme}
\textbf{Nota bene: } La velocità tangenziale($v$) indica quanto velocemente il punto si sposta sulla circonferenza(di raggio $r$); \\
La velocità angolare($\Omega$) indica quanto velocemente cambia l'angolo($\theta$) che il punto forma
\begin{gather*}
    \textbf{Velocità: } v = \frac{\Delta s}{\Delta t} = \frac{2 \pi r}{T} = \omega r = \sqrt{a_c r} \\
    \textbf{Raggio: } r = \frac{v T}{2 \pi} = \frac{v}{\omega} = \frac{v^2}{a_c} = \frac{a_c}{\omega^2} \\
    \textbf{Periodo: } T = \frac{2 \pi r}{v} = \frac{2 \pi}{\omega} \\
    \textbf{Velocità angolare: } \omega = \frac{v}{r} = \frac{2 \pi}{T} = \sqrt{\frac{a_c}{r}} \\
    \textbf{Accelerazione centripeta: } a_c = \frac{v^2}{r} = \omega^2 r \\
    \textbf{Posizione angolare: } \theta = \frac{s - s_0}{r} \\
    \textbf{Legge oraria} \\
    \textbf{Posizione angolare: } \theta(t) = \theta_0 + \omega t \\
    \textbf{Velocità angolare: } \omega = \frac{\theta - \theta_i}{t - t_i}
\end{gather*}
%%%%
\subsubsection{Moto circolare uniformemente accelerato(MCUA)}
\textbf{Nota bene: } L'accelerazione centripeta($\vec{a}_c$) permette al punto di mantenere la propria traiettoria sulla circonferenza. Cambia il verso, ma non il modulo della velocità($\vec{v}$) \\
L'accelerazione tangenziale($\vec{a}_T$, perpendicolare a $\vec{a}_c$) invece fa variare il modulo della velocità($\vec{v}$), è \textbf{costante}. \\
L'accelerazione totale($\vec{a}_{tot}$) è la risultante delle accelerazioni precedenti \\
\begin{gather*}
    \textbf{Accelerazione vett. tot.: } \vec{a}_{tot} = \vec{a}_T + \vec{a}_c \\
    \textbf{Accelerazione totale: } a_{tot} = \sqrt{a_T^2 + a_c^2} \\
    \textbf{Accelerazione angolare: } \alpha = \frac{d \omega}{d t} = \frac{\omega - \omega_0}{t - t_0} \\
    \textbf{Accelerazione tangenziale: } a_T = \alpha \cdot r \\
    \textbf{Legge oraria} \\
    \begin{cases}
        \theta = \theta_0 + \omega_0 t + \frac{1}{2} \alpha t^2 & \text{Posizione angolare} \\
        \omega = \omega_0 + \alpha t                            & \text{Velocità angolare}
    \end{cases} \\
    \textbf{Velocità angolare(senza t): } \\ \omega^2 = \omega_0^2 + 2 \alpha(\theta - \theta_0)
\end{gather*}
\subsubsection{Esempio}
\textbf{Nota bene: } per risolvere un problema del tipo: punto materiale con MCUA con $r = 1m$, $s_1 = 0.4m$, $t_1 = 2s$, $t_2 = 4s$, e $v_0 = 0.1 \frac{m}{s}$ dove chiede modulo dell'accelerazione totale al tempo $t_2$(quindi $a_{tot(2)} = \sqrt{a_T^2 + a_{c(2)}^2}$) possiamo impostare il seguente sistema:
\begin{gather*}
    \begin{cases}
        a_{tot(2)} = \sqrt{a_T^2 + a_{c(2)}^2} \\
        a_T = \frac{2(s - s_0 - v_0 t)}{t^2}   \\
        a_{c(2)} = \frac{v_2^2}{r}             \\
        v_{2} = v_0 + a_T \cdot t_2
    \end{cases}
\end{gather*}


\section{Dinamica}

\subsection{Principi fondamentali}
\begin{gather*}
    \text{Secondo principio: } \\
    \vec{F} = m \vec{a} \\
    \vec{a} = \frac{\vec{F}}{m} \\
    m = \frac{\vec{F}}{a} \\
    \text{Terzo principio: } \vec{F}_{AB} = -\vec{F}_{BA}
\end{gather*}

\subsection{Piano inclinato}
\begin{gather*}
    F_{P, x} = F_P \sin (\alpha) = m g \sin (\alpha) \\
    F_{P, y} = F_P \cos (\alpha) = m g \cos (\alpha)
\end{gather*}

\subsubsection{Piano inclinato senza attrito}
\includegraphics[width=0.75 \linewidth]{Dinamica/reazione-vincolare-nel-piano-inclinato.png} \\
\begin{gather*}
    \textbf{Accelerazione}: \begin{cases}
        a_y = 0 \\
        a_x = g \sin{\alpha}
    \end{cases}
\end{gather*}

\subsubsection{Piano inclinato con F verso l'alto}
\includegraphics[width=0.75 \linewidth]{Dinamica/variante-piano-inclinato-senza-attrito.png} \\
\textbf{Nota bene: } sull'asse y la forza($F$)che spinge l'oggetto verso l'alto non fa cambiare niente. Ricordiamo che abbiamo scelto \textbf{il piano inclinato} come asse del nostro sistema di riferimento.
\begin{gather*}
    \text{Forza risultante: } F_{ris} = F_{P, x} - F \\
    \text{Accelerazione: } a = \frac{F_{ris}}{m} \\
    \begin{cases}
        F_{P, x} > F & \text{Corpo scende, $a$ positiva} \\
        F_{P, x} < F & \text{Corpo sale, $a$ negativa}
    \end{cases}
\end{gather*}

\subsubsection{Piano inclinato con attrito}
\includegraphics[width=0.75 \linewidth]{Dinamica/piano-inclinato-con-attrito.png} \\
\begin{gather*}
    \textbf{Forza risultante: } F_{ris, x} = \sqrt{F_{P, x}^2 + F_{As}^2}
\end{gather*}
\textbf{Nota bene: } la forza d'attrito ($F_{A}$, attrito statico nell'immagine) ha verso opposto alla componente della forza peso sull'asse x($F_{P, x}$)
\subsubsection{Piano inclinato con attrito e forza aggiuntiva}
\includegraphics[width=\linewidth]{Dinamica/piano-inclinato-con-attrito-2.png} \\
\textbf{Primo caso: } forza risultante($F_{ris, x}$)tutta concentrata lungo l'asse $x$: $F_{ris, x} = F - F_A - F_{P, x}$ \\
\textbf{Secondo caso: } $F_{ris, x} = F + F_{P, x} - F_A$ \\

\subsubsection{Attrito Statico}
\begin{gather*}
    \textbf{Forza Attrito Statico: } \\ F_{As} = \mu_s \cdot F_\perp = \mu_s \cdot F_{P, y} = \mu_s m g \cos (\alpha) \\
    \textbf{Accelerazione: } \\ a = a_x = g \sin (\alpha) - \mu_s g \cos (\alpha) \\
    \textbf{Angolo critico per l'equilibrio: } \\ \alpha = \arctan (\mu_s) : \\
    \begin{cases}
        \text{Scivola}    & \text{Angoli } > \alpha  \\
        \text{Equilibrio} & \text{Angoli } <= \alpha
    \end{cases}    \\
    \textbf{Attrito Statico per Equilibrio: } \\ \mu_s = \tan (\alpha)
\end{gather*}
\textbf{Nota bene: } affinché una macchina(su terreno piano, $\cos (\alpha) = 0$) non slitti, la Forza di Attrito Statico($F_{As}$) deve essere uguale alla Forza($\vec{F}$) che la macchina esegue per andare avanti, quindi: $F_{As} = F \rightarrow \mu_s m g = m a \rightarrow \mu_s = \frac{a}{g}$
\subsubsection{Attrito Dinamico}
\textbf{Nota bene: } Per poter considerare il problema dal punto di vista dell'attrito dinamico($\mu_d$) il corpo deve essere già in movimento!
\begin{gather*}
    \textbf{Forza Attrito Dinamico:} \\ F_{Ad} = \mu_d \cdot F_\perp = \mu_d \cdot F_{P, y} = \mu_d m g \cos (\alpha) \\
    \textbf{Accelerazione: } \\ a = a_x = g \sin (\alpha) - \mu_d g \cos (\alpha)
\end{gather*}

\subsection{Molle e Forza Elastica}
\begin{center}
    \includegraphics[width=0.4 \linewidth]{Dinamica/forza-elastica.png}
\end{center}
\begin{gather*}
    \textbf{Lunghezza della molla a riposo: } L_0 \\
    \textbf{Elongazione: } x = L - L_0 \\
    \textbf{Legge di Hooke(Forza Elastica): } \\ F_e = -k x
\end{gather*}
\subsubsection{Oscillazione - Moto di una molla}
\textbf{Nota bene: } una molla quando viene rilasciata oscilla con un \textbf{Moto Armonico Semplice}.
\begin{gather*}
    \textbf{Frequenza angolare: } \omega = \sqrt{\frac{k}{m}} = \frac{2 \pi}{T} \\
    \textbf{Periodo: } \\ T = 2 \pi \sqrt{\frac{m}{k}} = \frac{2 \pi}{\omega} \\
    \textbf{Costante elastica: } \\ k = \frac{1}{m} (\frac{2 \pi}{T})^2 = m \omega^2 \\
    \textbf{Frequenza: } f = \frac{1}{t} = \frac{\omega}{2 \pi} \\
    \textbf{Ampiezza Oscillazione: } \\ A = \frac{x}{\cos (\omega T)} \\
    \textbf{Elongazione: } x = A \cos (\omega T) \\
    \textbf{Velocità oscillazione: } v = A \omega
\end{gather*}
\subsection{Forza Centripeta}
\textbf{Nota bene: } possiamo ricondurci alla forza centripeta($F_c$) usando le formule per la forza($F = m \cdot a$) e l'accelerazione centripeta($a_c = \frac{v^2}{r} = \omega^2 \cdot r $)! Utilizziamo queste formule per problemi come \textbf{macchine in un circuito di raggio $r$ e coefficiente di attrito dinamico $\mu_d$ dove non deve slittare}.
\\
\begin{gather*}
    \textbf{Forza centripeta: } \\ F_c = m \frac{v^2}{r} cos (\alpha) \\
    \textbf{Forza centripeta: } \\ F_c = m \cdot \omega^2 \cdot r
\end{gather*}

\textbf{Nota bene: } se vogliamo che un corpo rimanga nella sua traiettoria circolare, allora la la forza di accelerazione($F_a$)deve essere uguale alla forza centripeta($F_c$).

\begin{center}
    \includegraphics[width=0.7 \linewidth]
    {Dinamica/forza-centripeta.png}
\end{center}
\begin{gather*}
    \textbf{Equivalenza: } F_a = F_c \rightarrow \mu_d  g = \frac{v^2}{r} \\
    \textbf{Coefficiente d'attrito minimo: } \\ \mu = \frac{v^2}{g \cdot r} \\
    \textbf{Velocità minima: } v = \sqrt{\mu_d \cdot g \cdot r}
\end{gather*}
\subsection{Lavoro}

\textbf{Nota bene: } il lavoro($F$) è energia trasferita ad un corpo mediante le forze che agiscono su di esso. \\
\begin{center}
    \includegraphics[width=0.4 \linewidth]{Dinamica/Lavoro/il-lavoro-di-una-forza.png}
\end{center}
\begin{gather*}
    \textbf{Lavoro: } L = \vec{F} \cdot \vec{s} = F \cdot  s
\end{gather*}

\subsection{Lavoro della forza peso}

\begin{gather*}
    \textbf{Lavoro: } L = -mg(y_{finale} - y_{iniziale})
\end{gather*}

\subsection{Lavoro della forza elastica}

\begin{gather*}
    \textbf{Lavoro: } L = - \frac{1}{2} k (x_f^2 - x_i^2) \\
    \textbf{Elongazione iniziale: } x_i = L_i - L_0 \\
    \textbf{Elongazione finale: } x_f = L_f - L_0
\end{gather*}

\textbf{Nota bene: } nel caso di una molla a riposo($x_i = 0$)il lavoro($L$) sarà sempre negativo, che la molla venga compressa($x_f < 0$) o che venga allungata($x_f > 0$), poiché la forza elastica della molla resiste all'elongazione.

\begin{gather*}
    \textbf{Molla compressa: } x_f < 0 \\
    \textbf{Molla allungata: } x_f > 0 \\
    \textbf{Lavoro: } L = - \frac{1}{2} k (x_f^2)
\end{gather*}
\subsection{Energia}
\textbf{Nota bene: } l'energia è una grandezza che esprime la capacità di un corpo/sistema di compiere un lavoro, indipendentemente dal fatto che il lavoro venga compiuto o meno. Si divide in: \textit{cinetica, potenziale, meccanica}.

\subsection{Energia Cinetica}

\begin{gather*}
    \textbf{Teorema Energia Cinetica: } L = \Delta K \\
    \textbf{Energia Cinetica: } K = \frac{1}{2} m v^2 \\
    \textbf{Massa: } m = \frac{2K}{v^2} \\
    \textbf{Velocità: } v = \sqrt{\frac{2K}{m}} \\
    \textbf{Variazione Energia Cinetica: } \\ \Delta K = K_f - K_i = \frac{1}{2} m v_f^2 - \frac{1}{2} m v_i^2
\end{gather*}

\subsection{Energia potenziale}

\begin{gather*}
    \textbf{Energia Potenziale: } \Delta U = U_f - U_i = -L \\
    \textbf{Solo Forze conservative: } \Delta K = - \Delta U \\
    \textbf{En. Pot. Gravitazionale: } \\
    U = m g h \quad
    m = \frac{U}{gh} \quad
    h = \frac{U}{mg} \\
    \textbf{Energia Potenziale Elastica: } \\
    U = \frac{1}{2} k x^2 \quad
    k = \frac{2U}{x^2} \quad
    x = \sqrt{\frac{2U}{k}}
\end{gather*}

\subsection{Energia meccanica}

\begin{gather*}
    \textbf{Energia Meccanica: } E = K + U \\
\end{gather*}
\subsubsection{Conservazione Dell'Energia Meccanica}
\textbf{Nota bene: } applicabile solo quando le forze in gioco sono conservative(\textit{Forza Peso, Forza Elastica}).
\begin{gather*}
    \textbf{Conservazione dell'energia: } \\ K_i + U_i = K_f + U_f \rightarrow E_f = E_i
\end{gather*}

\subsubsection{Non Conservazione Dell'Energia Meccanica}

\textbf{Nota bene: } applicabile solo quando una delle forze in gioco è non conservativa(\textit{Forza D'Attrito}). Il lavoro che troviamo è pari all'energia mancante dal sistema(a causa di una forza dissipativa). \\
In generale: \textbf{Lavoro Forza Non Conservativa = Variazione Energia Meccanica}.

\begin{gather*}
    \textbf{Lavoro Attrito: } \\ L_A = \Delta E = \\ = \Delta K - \Delta U = \\ \frac{1}{2} m v_f^2 - \frac{1}{2} m v_i^2 - mgh_f + mgh_i \\
    \textbf{Lavoro Attrito: } \\ L_A = F_{A} \cdot s = \mu \cdot s \cdot m g \cos (\alpha)
\end{gather*}

\subsection{Oscillatore armonico(Pendolo)}
\begin{gather*}
    \textbf{Frequenza: } T = 2 \pi \sqrt{\frac{l}{a_g}} \\
    \textbf{Lunghezza Pendolo: } l = \frac{g T^2}{4 \pi^2} \\
    \textbf{Accelerazione gravitazionale: } \\ a_g = \frac{M \cdot G}{r_{distanza}^2}
\end{gather*}


\section{Gravitazione}

\begin{gather*}
    \textbf{Accelerazione gravitazionale: } \\ a_g = \frac{M\cdot G}{r^2} \\
    F = M \cdot a_g \\
    \textbf{Velocità Centripeta: } v_c= \sqrt{\frac{G \cdot M}{r_{orbita}}} \\
    \textbf{Velocità di Fuga: } v_{fuga}=\sqrt{\frac{2 \cdot G \cdot M}{r_{orbita}}} \\
    \textbf{Periodo di Rivoluzione}: \\ T^2 = \frac{4 \pi^2 r^3_{orbita}}{G \cdot M} \\ T = \frac{2 \pi r}{v_{}}\\
    \textbf{Legge di gravitazione di Newton}: \\ F=F_1=F_2 = G\cdot\frac{M_1\cdot M_2}{r^2}
\end{gather*}


\section{Fluidodinamica}

\includegraphics[width=1 \linewidth]{FluidoDinamica/flusso.png} \\

\begin{gather*}
    \textbf{Densità(massa volumica): } \\ \rho = \frac{m}{V} \quad m = \rho V \quad V = \frac{m}{\rho} \\
    \textbf{Pressione: } p = \frac{F_\perp}{S} \\
    \textbf{Pressione Fluidi Comprimibili: } \\ p = p_0e^{-\frac{\rho_0 g}{p_0}z} \\
    \textbf{Pressione atm. ad altitudine $z$: } \\
    p = p_0e^{-\frac{z}{8006km}} \\
    \textbf{Legge di Stevino: } p = \rho g h \\
    \textbf{Principio di Pascal: } \\ p_1 = p_2 \\ \frac{F_1}{S_1} = \frac{F_2}{S_2} \\
    \textbf{Principio di Archimede: } \\ F_A = P_{fl} = m_{fl} g = \rho_{fl} V_{imm} g \\ \begin{cases}
        \rho_{corpo} > \rho_{fluido} & \text{corpo affonda}       \\
        \rho_{corpo} < \rho_{fluido} & \text{corpo in equilibrio} \\
        \rho_{corpp} > \rho_{fluido} & \text{corpo galleggia}
    \end{cases} \\
    \textbf{Formula del galleggiamento: } \\ F_A = F_P \\ \rho_{liq} V_{imm} = \rho_{corpo} V_{tot} \\
    \textbf{Portata: } \\ Q = \frac{V}{\Delta t} \\ Q = A \cdot v \\
    \textbf{Teorema di Bernoulli: } \\ p_1 + \frac{1}{2} \rho v_1^2 + \rho g h_1 = p_2 + \frac{1}{2} \rho v_2^2 + \rho g h_2 \\
    \textbf{Equazione di Continuità(ideale): } \\  Q = v_1 S_1 = v_2 S_2 \\
    \textbf{Equazione di Continuità(reale): } \\ Q = \rho_1 v_1 S_1 = \rho_1 v_2 S_2 \\
    \textbf{Velocità 1: } v_1 = \frac{Q}{S_1} = \frac{S_2}{S_1} v_2 \\
    \textbf{Velocità 2: } v_2 = \frac{Q}{S_2} = \frac{S_1}{S_2} v_1 \\
    \textbf{Sezione 1: } S_1 = \frac{v_2}{v_1} S_2 \\
    \textbf{Sezione 2: } S_2 = \frac{v_1}{v_2} S_1 \\\
    \textbf{Differenza di pressione: } \\
    \Delta p = \frac{1}{2} \rho (v_2^2 - v_1^2) \\
    \Delta p = \frac{1}{2} \rho (\frac{Q^2}{S_2^2} - \frac{Q^2}{S_1^2})
\end{gather*}

\section{Termodinamica}

\begin{gather*}
    \textbf{Legge dei gas perfetti:} \\
    P \cdot V = n_{moli} \cdot R \cdot T \\
    R_{costante} = 8,31 \frac{J}{mol \cdot K} \\
    \textbf{Volume molare a $0^\circ$: } \\
    V_m = \frac{1 \textit{mol} \cdot 0,0821 \frac{l \cdot atm}{mol \cdot K} \cdot 273 K}{1 \textit{atm}} =  22,414 \ell \\
    \textbf{$1^a$ legge della Termodinamica:} \\
    Q + L = \Delta E_{int}  \\
    \textbf{Capacità termica: } C = \frac{Q}{\Delta T} \\
    \textbf{Calore specifico: } c = \frac{C}{m} \\
    \textbf{Lavoro svolto da un gas: } L = - \int p \cdot dV \\
    \textbf{Trasf. isocora (V const): } L = 0 \\
    \textbf{Trasf. isobara (p const): } L = -p(V_f - V_i)    \\
    \textbf{Trasf. isoterma (T const): } L = -nRT \cdot \ln \frac{V_f}{V_i}
\end{gather*}


\section{Circuiti}
\textbf{Nota bene: } la \textbf{differenza di potenziale/tensione($V$)} in parallelo non cambia, così come la \textbf{corrente($I$)} in serie non cambia.
\begin{gather*}
    \textbf{Differenza di Potenziale: } \\ V = I \cdot R \quad I = \frac{\Delta V}{R} \\
    \textbf{Potenza Dissipata: } \\
    W = I \cdot V = I^2 R = \frac{V^2}{R} \\
    \textbf{Condensatori: } \\ C = \frac{Q}{\Delta V}
\end{gather*}
\subsection{Leggi di Kirchhoff}
\begin{gather*}
    \includegraphics[width=0.7 \linewidth]{Circuiti/node_law.png} \\
    I_1 = I_3 + I_4 \\
    \textbf{Resistenze in serie: } \\
    \includegraphics[width=0.75 \linewidth]    {Circuiti/resistenze_in_serie.png} \\
    R_{12} = R_1 + R_2 \\
    \includegraphics[width=0.75 \linewidth]    {Circuiti/resistenze_in_parallelo.png} \\
    \frac{1}{R_{123}} = \frac{1}{R_1} + \frac{1}{R_2} + \frac{1}{R_3} \\ \text{Oppure: } R_{123} = \frac{R_1 R_2 R_3}{R_1 + R_2 + R_3} \\
    \textbf{Condensatori in serie: } \\
    \includegraphics[width=0.75 \linewidth]    {Circuiti/condensatori_in_serie.png} \\
    \frac{1}{C_{123}} = \frac{1}{C_1} + \frac{1}{C_2} + \frac{1}{C_3} \\
    \text{Oppure: } C_{123} = \frac{C_1 C_2 C_3}{C_1 + C_2 + C_3} \\
    \includegraphics[width=0.75 \linewidth]    {Circuiti/condensatori_in_parallelo.png} \\
    C_{123} = C_1 + C_2 + C_3
\end{gather*}

\end{multicols}

\end{document}
