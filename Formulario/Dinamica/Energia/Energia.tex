\subsection{Energia}
\textbf{Nota bene: } l'energia è una grandezza che esprime la capacità di un corpo/sistema di compiere un lavoro, indipendentemente dal fatto che il lavoro venga compiuto o meno. Si divide in: \textit{cinetica, potenziale, meccanica}.

\subsection{Energia Cinetica}

\begin{gather*}
    \text{Teorema Energia Cinetica: } L = \Delta K \\
    \text{Energia Cinetica: } K = \frac{1}{2} m v^2 \\
    \text{Massa: } m = \frac{2K}{v^2} \\
    \text{Velocità: } v = \sqrt{\frac{2K}{m}} \\
    \text{Variazione Energia Cinetica: } \\ \Delta K = K_f - K_i = \frac{1}{2} m v_f^2 - \frac{1}{2} m v_i^2
\end{gather*}

\subsection{Energia potenziale}

\begin{gather*}
    \text{Energia Potenziale: } \Delta U = U_f - U_i = -L \\
    \text{Solo Forze conservative: } \Delta K = - \Delta U \\ 
    \text{En. Pot. Gravitazionale: } \begin{cases}
        U = m g h \\
        m = \frac{U}{gh} \\
        h = \frac{U}{mg}
    \end{cases} \\
    \text{Energia Potenziale Elastica: } \begin{cases}
        U = \frac{1}{2} k x^2 \\
        k = \frac{2U}{x^2} \\
        x = \sqrt{\frac{2U}{k}}
    \end{cases}
\end{gather*}

\subsection{Energia meccanica}

\begin{gather*}
    \text{Energia Meccanica: } E = K + U \\
\end{gather*}
\subsubsection{Conservazione Dell'Energia Meccanica}
\textbf{Nota bene: } applicabile solo quando le forze in gioco sono conservative(\textit{Forza Peso, Forza Elastica}).
\begin{gather*}
    \text{Conservazione dell'energia(cinetica): } \\ K_i + U_i = K_f + U_f \rightarrow E_f = E_i \\ \frac{1}{2} m v_i^2 + mgh_i = \frac{1}{2} mv_f^2 + mgh_f \\
    \text{Conservazione dell'energia(elastica): } \\ K_i + U_i = K_f + U_f \rightarrow E_f = E_i \\ \frac{1}{2} m v_i^2 + \frac{1}{2}kx^2 = \frac{1}{1}mv_f^2 + mgh_f \\
    \textbf{Nota bene: }\text{Generalmente $v_i = 0 \frac{m}{s}$, quindi: } \\
    \begin{cases}
        \frac{1}{2}kx^2 = \frac{1}{1}mv_f^2 + mgh_f \\
        v = \sqrt{\frac{k x^2}{m} - 2gh}
    \end{cases}
\end{gather*}

\subsubsection{Non Conservazione Dell'Energia Meccanica}

\textbf{Nota bene: } applicabile solo quando una delle forze in gioco è non conservativa(\textit{Forza D'Attrito}). Il lavoro che troviamo è pari all'energia mancante dal sistema(a causa di una forza dissipativa). \\
In generale: \textbf{Lavoro Forza Non Conservativa = Variazione Energia Meccanica}.

\begin{gather*}
    \text{Lavoro Attrito: } \\ L_A = \Delta E = \\ = \Delta K - \Delta U = \\ \frac{1}{2} m v_f^2 - \frac{1}{2} m v_i^2 - mgh_f + mgh_i \\
    \text{Lavoro Attrito: } \\ L_A = F_{A} \cdot s = \mu \cdot s \cdot m g \cos (\alpha)
\end{gather*}