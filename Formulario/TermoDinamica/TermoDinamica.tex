\section{Termodinamica}

\begin{gather*}
    \textbf{Legge dei gas perfetti:} \\
    P \cdot V = n_{moli} \cdot R \cdot T \\
    R_{costante} = 8,31 \frac{J}{mol \cdot K} \\
    \textbf{Volume molare a $0^\circ$: } \\
    V_m = \frac{1 \textit{mol} \cdot 0,0821 \frac{l \cdot atm}{mol \cdot K} \cdot 273 K}{1 \textit{atm}} =  22,414 \ell \\
    \textbf{$1^a$ legge della Termodinamica:} \\
    Q + L = \Delta E_{int}  \\
    \textbf{Capacità termica: } C = \frac{Q}{\Delta T} \\
    \textbf{Calore specifico: } c = \frac{C}{m} \\
    \textbf{Lavoro svolto da un gas: } L = - \int p \cdot dV \\
    \textbf{Trasf. isocora (V const): } L = 0 \\
    \textbf{Trasf. isobara (p const): } L = -p(V_f - V_i)    \\
    \textbf{Trasf. isoterma (T const): } L = -nRT \cdot \ln \frac{V_f}{V_i}
\end{gather*}
